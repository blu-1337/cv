% Copyright 2009-2011,2015-2019 Dominik Wagenfuehr <dominik.wagenfuehr@deesaster.org>
% Dieses Dokument unterliegt der Creative-Commons-Lizenz
% "Namensnennung-Weitergabe unter gleichen Bedingungen 4.0 International"
% [http://creativecommons.org/licenses/by-sa/4.0/deed.de].
%
% Beispiel-Bewerberfoto: Copyright 2013 TVJunkie
% https://commons.wikimedia.org/wiki/File:Gnome-Wikipedia-user-female.png
% Lizenz: Creative-Commons-Lizenz "Attribution-Share Alike 3.0 Unported"
% [https://creativecommons.org/licenses/by-sa/3.0/deed.en]

\documentclass[fontsize=12pt,parskip=half-]{scrartcl}

% Liest die vordefinierten Befehle ein. Nicht veraendern!
\usepackage{bewerbung-latex}

% Optionales Paket, falls jemand \Telefon und \Email
% als Symbol nutzen will.
%\usepackage{marvosym}

%%%%%%%%%%%%%%%%%%%%%%%%%%%%%%%%%%%%%%%%%%%%%%%%%%%
% Ab hier sollte man selbst Aenderungen vornehmen.
%%%%%%%%%%%%%%%%%%%%%%%%%%%%%%%%%%%%%%%%%%%%%%%%%%%

% Andere Schrift festlegen (nur machen, falls man die korrekten Namen kennt).
% Das erste Argument ist der Paketname für pdfTeX, das zweite Argument
% der Schriftname für XeTeX und LuaTeX.
% Als Standard wird hier TeX Gyre Pagella benutzt (Paket "tex-gyre").
\SetzeSchrift{tgpagella}{TeX Gyre Pagella}     % optional

% Umschalten auf serifenlose Schrift. Wenn gewuenscht, einkommentieren.
% Achtung: Nicht jede serifenlose Schrift kommt mit den Kapitaelchen in
% den Ueberschriften (Lebenslauf, Zu meiner Person etc.) klar. Sollten
% diese mit Serifen angezeigt werden, dann muss man zusaetzlich das
% optionale Argument [noscshape] angeben. Dann gibt es natuerlich auch
% keine Kapitaelchen!
%\NutzeSerifenlosenSchrift[]{}%

% Zuerst die Sprache festlegen.
% Per Standard ist "Deutsch" gesetzt.
% Zusaetzliche, moegliche Angaben sind: "Schweiz", "British", "American"
\SetzeSprache{Deutsch}          % optional

% Eigene Daten festgelegt.
\VollerName{Cristian Gîlcă}
\AbsenderStrasse{Str. Cerna Nr. 15A, Bl. a5. Sc. 2, Ap. 1}
\AbsenderPLZOrt{200185 Craiova}
\AbsenderLand{Rumänien}
% Als erstes Argument kann man beim Telefon eine Angabe
% davor angeben. Der Wert darf auch leer sein.
% Bei Telefonnummer ggf. Laenderkennung ergaenzen, wenn
% die Telefonnummer nicht im Land des Empfaengers ist.
\bwTelefon{}{+40 727 292 100}
%\bwTelefon{\Telefon}{+49 170 567890}
% Als erstes Argument kann man bei der E-Mail eine Angabe
% davor angeben (z.B. "E-Mail:". Der Wert darf auch leer sein.
% Zusaetzlich kann optional eine Farbe angegeben werden,
% mit der die Mailadresse dargestellt werden soll.
% Standard ist "black"  als Farbe. Beispiel: \EMail[orange]{...}{...}
% Achtung: Die Farbe wird fuer alle Links benutzt, damit es einheitlich aussieht.
\bwEMail[Blue]{}{gilca.cristian@gmail.com}
%\bwEMail{\Email}{eva.mustermann@musterstadt.de}
\Ort{Craiova}
\Datum{\HeutigerTag}
\Geburtstag{January 1, 1985}
\Geburtsort{Berlin}
% Der Ausbildungsgrad und die Details sind optional.
% Leer lassen oder auskommentieren, wenn nichts angezeigt werden soll.
\Details{single, travel flexible} % verheiratet, 2 Kinder, etc.
\Ausbildungsgrad{Doctor of Veterinary Medicine}    % optional
% Optionales Xing- und Linkedin-Profil
% Beides wird nur auf der Persoenlichen Seite angezeigt,
% nicht im Anschreiben.
\XingProfil{Eva\_Mustermann}
\LinkedInProfil{evamustermann}

% Die Unterschrift sollte als Bilddatei vorliegen.
% Leer lassen oder auskommentieren, wenn keine Signatur eingebunden
% werden soll.
% Als Option kann man die Breite der Unterschrift angeben.
% Eine Breite von 5cm ist der Standard.
\UnterschriftenDatei[3.9cm]{signatur-bewerber.png}           % optional

% Das Bewerberfoto.
% Leer lassen oder auskommentieren, wenn keine Foto auf
% der "Meine Seite"-Seite angezeigt werden soll.
% Als Option gibt man die Hoehe des Bildes an. Eine Hoehe von 10.5 cm
% ist das groesste, was noch auf die Seite passt.
\BewerberFoto[6cm]{foto-bewerber.jpg}                 % optional

% Adressat festlegen
% Die Abteilung und der Vorname sind optional und koennen leer
% gelassen oder auskommentiert werden.
% Wenn der Nachname leergelassen oder auskommentiert wird,
% wird die Anrede automatisch durch "Damen und Herren" ersetzt.
\Firma{Capgemini Engineering}
\Abteilung{Altran Deutschland S.A.S. \& Co. KG}            % optional
%\AdressatVorname{Katrin }              % optional
%\AdressatNachname{Kraus}                % ggf. optional
%\AnschriftStrasse{Niederlassung Karlsruhe}
%\AnschriftPLZOrt{Lorenzstr. 29 \newline
%	76135 Karlsruhe
%}

% Als optionales Argument hat der Titel eine Kurzform, die nur im Briefkopf
% benutzt wird. Wenn das optionale Argument fehlt, wird die normale Angabe 
% fuer Briefkopf und Anrede benutzt.
% Achtung: Im Englischen ersetzt ein Titel die Anredeform.
% Daneben werden im Britischen keine Punkte gesetzt (muss man selbst dran denken!)
%\AdressatTitel[Dr.]{Doktor}   % optional

% Achtung: Anrede muss am Ende dieser Liste stehen,
% damit dies korrekt um Titel und Name ergaenzt wird!
% Deutsch/Schweiz: Frau, Herr oder leer lassen
% British/American: Mr, Mrs, Ms oder leer lassen
% Achtung: Im Amerikanischen wird automatisch ein Punkt ergaenzt.
% Achtung: Im Englischen ersetzt ein Titel die Anredeform.
\Anrede{Frau}

% Man kann hier einen zweiten Ansprechpartner beim Anschreiben angeben.
% Die Adresszeile wird entsprechend ergaenzt.
%\ZweiterAdressatVorname{Pitti}            % optional
%\ZweiterAdressatNachname{Platsch}         % ggf. optional
%\ZweiterAdressatTitel[Dr]{Doctor}         % optional
%\ZweiterAdressatAnrede{Mr}

% Die Stelle, auf die man sich bewirbt
\Bewerberstelle{Junior Cloud \& DevOps Engineer Automotive (m/w/d)}

% Beginn des Dokuments, nicht aendern!
\begin{document}

%%%%%%%%%%%%%%%%%%%%%%%%%%%
% Das Anschreiben
%%%%%%%%%%%%%%%%%%%%%%%%%%%

% Anschreiben und Motivationsseite sind per Standard als linksbuendiger
% Flattersatz gesetzt. Mit der nachfolgenden Option kann man auf
% Blocksatz umstellen.
%\NutzeBlocksatz{}

% Ausrichtung des Absenders im Anschreiben (nur im Anschreiben, nicht
% auf der eigenen Seite!)
% Aufgrund der Ausrichtung des Adressaten will man ggf. auch den Absender
% neu ausrichten, damit es nicht direkt übereinander steht. Per Standard:
% Deutsch: rechts
% Schweiz: links
% Aber man kann es manuell umschalten, wenn man will.
%\AbsenderAusrichtung{rechts}          % optional

% Ausrichtung des Adressaten im Adressfeld
% Es gibt Laender, wo das Adressfeld nicht links,
% sondern rechts steht. Per Standard gilt:
% Deutsch: links
% Schweiz: rechts
% Aber man kann es manuell umschalten, wenn man will.
%\AddressatAusrichtung{links}         % optional

% Abstand zwischen dem Absender und dem Adressat im Anschreiben.
% Gemessen wird in Zeilen, d.h. der Wert 1.5 steht fuer anderthalb Zeilen.
% Der Standard-Wert ist 0.
%\AbstandZwischenAdressen{0}      % optional

% Abstand vor dem eigentlichen Anschreiben (inkl. Ort und Datum).
% Gemessen wird in Zeilen, d.h. der Wert 1.5 steht fuer anderthalb Zeilen.
% Der Standard-Wert ist 1.
\AbstandVorAnschreiben{0}      % optional

% Abstand zwischen dem Ort und Datum und der Betreffzeile.
% Gemessen wird in Zeilen, d.h. der Wert 1.5 steht fuer anderthalb Zeilen.
% Der Standard-Wert ist 1.
\AbstandZwischenDatumUndBetreff{1}     % optional

% Anschreibenseite vergroessern, d.h. es ist damit moeglich, ueber den
% eigentlichen unteren Rand zu schreiben, falls das Anschreiben etwas
% laenger geworden ist.
% Gemessen wird in Zeilen, d.h. der Wert 1.5 steht fuer anderthalb Zeilen.
% Der Standard-Wert ist 0. Als Maximalwert sollte man 3 einstellen, ansonsten
% wirkt das Anschreiben vom Aufbau sehr unausgeglichen.
\AnschreibenSeiteVergroessern{0}     % optional

% Abstand zwischen Unterschrift und den Anlagen im Anschreiben
% Gemessen wird in Zeilen, d.h. der Wert 1.5 steht fuer anderthalb Zeilen.
% Der Standard-Wert ist 1.
\AbstandVorAnlagen{2}     % optional

% Hinweis auf Anlagen
% Wenn nicht benoetigt, dann einfach auskommentieren.
%\AnschreibenAnlage{Anlagen}

\begin{Anschreiben}
    % Hinweis Anfang – nach dem Lesen loeschen!
    % In der Schweiz beginnt man den Satz im Uebrigen gross, da kein
    % Komma bei der Anrede benutzt wird. Im Englischen faengt man auch gross an.




ich bin auf Ihr aktuelles Jobangebot auf Ihre Webseite capgemini.com aufmerksam geworden. Deswegen möchte ich mich gerne hiermit für die Junior Position im Bereich DevOps bei Ihnen bewerben. 

Ich habe 2021 mein Studium der Informatik an der TU Ilmenau in Deutschland abgeschlossen. Danach habe ich ein Jahr lang als Configuration Manager bei Hella KGaA Hueck \& Co. in Rumänien gearbeitet. Dies hat mein Interesse an der DevOps-Seite des Betriebs geweckt, da ich meine Programmierkenntnisse nutzen möchte, um IT-Systeme besser zum Laufen zu bringen. 

Gern möchte ich mich weiterentwickeln und praxisnahe Programmiererfahrung bei Ihrer Firma sammeln. Durch eine vorherige Tätigkeit als Configuration Manager, Aushilfskraft und Webentwickler verfüge ich auch über sehr gute Programmierkenntnisse. 

Weitere Informationen zu meinem bisherigen Werdegang entnehmen Sie bitte dem angefügten Lebenslauf. 
    % Hinweis Ende – nach dem Lesen loeschen!
\end{Anschreiben}

% Optional kann man die Farbe für die Ueberschriften ab hier festlegen.
% Dazu zaehlt auch die eigene Seite.
% Als Optionen kann man die normalen LaTeX-Farben verwenden (Englisch):
% https://en.wikibooks.org/wiki/LaTeX/Colors
% Per Standard wird "Black" (Schwarz) verwendet.
% Achtung: Man sollte es nicht zu bunt treiben!
%\UeberschriftFarbe{Black}            % optional



%%%%%%%%%%%%%%%%%%%%%%%%%%%%%%%%%%%%%%%%%%%
% Lebenslauf
%%%%%%%%%%%%%%%%%%%%%%%%%%%%%%%%%%%%%%%%%%%

% Ausrichtung der Ueberschriften.
% Man kann die Ausrichtung prinzipiell vor jedem Kapitel
% (Lebenslauf, Motivation, Anlagen) neu einstellen, aber gleichförmiger
% ist es, wenn man es nur einmal hier definiert.
% Moegliche Angaben sind: links, rechts, mittig.
% Per Standard wird "rechts" ausgewaehlt.
\UeberschriftAusrichtung{rechts}         % optional

% Schriftgroesse der Ueberschriften
% Erlaubt sind alle Standard-Schriftgroessenangaben.
% Sinnvoll sind \LARGE, \huge, \Huge
% Standard ist \LARGE
\UeberschriftGroesse{\LARGE}            % optional

% Bezeichnung des Lebenslaufes.
% Per Standard gilt:
% Deutsch/Scheiz: "Lebenslauf"
% American: "Resumé"
% British: "Curriculum Vitae"
%\UeberschriftLebenslauf{Lebenslauf}     % optional

% Der Lebenslauf ist in Abschnitte unterteilt. Jeder Abschnitt hat dabei
% einen Titel. Innerhalb des Abschnitts (zwischen "begin" und "end")
% legt man mit \EintragCV einen Eintrag an.
% In der Textgestaltung ist man dabei ansonsten frei.
% Legt den Einschub des Inhalts eines CV-Abschnitte (d.h. der Tabelle) fest.
% Standard ist 8pt.
% Das optionale Argument "buendig" gibt an, dass die Ueberschrift zusaetzlich
% noch buendig zur Tabelle sein soll und nicht linksbuendig per Standard.
\EinschubCV[buendig]{8pt}            % optional
%\EinschubCV{8pt}

% Optional kann man die Farbe fuer die einzelnen Abschnitte
% und/oder fuer die Linien darunter festlegen.
% Als Optionen kann man die normalen LaTeX-Farben verwenden (Englisch):
% https://en.wikibooks.org/wiki/LaTeX/Colors
% Per Standard wird "Black" (Schwarz) verwendet.
% Achtung: Man sollte es nicht zu bunt treiben!
\AbschnittFarbe{Blue}                  % optional
\AbschnittLinienFarbe{Blue}            % optional

% Ausrichtung der Unterschrift im Lebenslauf.
% Moegliche Angaben sind: links oder rechts (mittig finde ich zu aussergewoehnlich)
% Per Standard wird "links" ausgewaehlt.
\LebenslaufUnterschriftAusrichtung{links}         % optional


% Man kann die Anlagen auch direkt in das Dokument einbinden,
% wenn sie als PDF vorliegen. Oder man versendet sie separat.
% Mit der Option "quer" wird die Anlage im Querformat eingebunden.
%\AnlageEinfuegen{diplomurkunde.pdf}
%\AnlageEinfuegen[quer]{doktorurkunde.pdf}

% Ende des Dokuments, nicht aendern!
\end{document}
